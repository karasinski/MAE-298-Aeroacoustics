\documentclass[onecolumn,10pt]{jhwhw}

\usepackage{epsfig} %% for loading postscript figures
\usepackage{amsmath}
\usepackage{graphicx}
\usepackage{grffile}
\usepackage{pdfpages}
\usepackage{algpseudocode}
\usepackage{wrapfig}
\usepackage{booktabs}
\usepackage{multicol}

% Default fixed font does not support bold face
\DeclareFixedFont{\ttb}{T1}{txtt}{bx}{n}{12} % for bold
\DeclareFixedFont{\ttm}{T1}{txtt}{m}{n}{12}  % for normal

% Custom colors
\usepackage{color}
\usepackage{listings}
\usepackage{framed}
\usepackage{caption}
\usepackage{bm}
\captionsetup[lstlisting]{font={small,tt}}

\author{John Karasinski}
\title{EAE 298 Aeroacoustics \\ Fall Quarter 2016 \\ Homework \#3}

\begin{document}
\maketitle

\problem{[50 points]}
The acoustic wave equation without considering the source is expressed as follows:
\begin{align*}
\dfrac{1}{c^2} \dfrac{\partial^2 p}{\partial t^2} - \nabla^2 p = 0
\end{align*}
We can define a new function $\widetilde{p}$ using the imbedding technique as follows:
\begin{align*}
\widetilde{p} = p, & \hspace{1em} f > 0 \\
\widetilde{p} = 0, & \hspace{1em} f < 0
\end{align*}
where $f=0$ describes the arbitrary moving body. Show that the wave equation whose sound is generated by an arbitrary moving body (=0) can be expressed as follows:
\begin{align*}
\dfrac{1}{c^2} \dfrac{\partial^2 \widetilde{p}}{\partial t^2}
- \overline{\nabla^2} \widetilde{p}
= - \left [ \dfrac{M_n}{c} \dfrac{\partial p}{\partial t} + p_n \right] \delta (f)
- \dfrac{1}{c} \dfrac{\partial}{\partial t} \left[ M_n p \delta(f) \right]
- \nabla \cdot \left[ p \vec{n} \delta (f) \right]
\end{align*}
where $\vec{n}$ is the unit normal vector on the surface and $p_n = \nabla p \cdot \vec{n}$. Now we can use the Green’s function of the wave equation in the unbounded space, the so-called free-space Green’s function, to find the unknown function $p(\vec{x}, t)$ everywhere in space. The result is the Kirchhoff formula for moving surfaces.

\problem{[50 points]}
Farassat’s formulation 1 for the loading noise is given as
\begin{align*}
4 \pi p_L^{\prime}(\vec{x}, t) = \dfrac{1}{c} \dfrac{\partial}{\partial t}
\int_{f=0} \left[ \dfrac{L_r}{r(1-M_r)} \right]_{ret} dS + \int_{f=0} \left[ \dfrac{L_r}{r^2(1-M_r)} \right]_{ret} dS
\end{align*}
where $L_r = \Delta P \vec{n} \cdot \hat{r} = \Delta P \cos \theta$. This formulation 1 is difficult to compute since the observer time differentiation is outside the integrals. A much more efficient and practical formulation can be derived by carrying the observer time derivate inside the integrals (formulation 1A). Show that formulation 1A for the loading noise becomes
\begin{align*}
4 \pi p_L^{\prime}(\vec{x}, t) =& \dfrac{1}{c} \int_{f=0} \left[ \dfrac{\dot{L_r}}{r(1-M_r)^2} \right]_{ret} dS \\
&+ \int_{f=0} \left[ \dfrac{L_r-L_M}{r^2(1-M_r)^2} \right]_{ret} dS
+ \dfrac{1}{c} \int_{f=0} \left[ \dfrac{L_r (r \dot{M_r} + c(M_r-M^2))}{r^2(1-M_r)^3} \right]_{ret} dS
\end{align*}
where $L_M = \vec{L} \cdot \vec{M}$.

\end{document}
